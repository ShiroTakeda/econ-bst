%#!pdflatex 
%#BIBTEX bibtex econ-example
% Filename:            econ-example.tex
% Author:              Shiro Takeda
% First-written:       <2002/11/02>
%
% This document explains how to use econ.bst.

%############################## Main #################################

\documentclass[10pt]{article}

%% Use natbib.sty.
\usepackage{natbib}
\usepackage{fancybox}
\usepackage{newtxtext,newtxmath}
\usepackage{url}
\usepackage{graphicx}
\usepackage{color}
\definecolor{MyBrown}{rgb}{0.3,0,0}
\definecolor{MyBlue}{rgb}{0,0.2,0.6}
\definecolor{MyRed}{rgb}{0.4,0,0.1}
\definecolor{MyGreen}{rgb}{0,0.4,0}
\usepackage[bookmarks=true,bookmarksnumbered=true,colorlinks=true,linkcolor=MyBlue,citecolor=MyRed,filecolor=MyBlue,urlcolor=MyGreen]{hyperref}

\setlength{\topmargin}{-15pt} 
\setlength{\textheight}{638pt}
\setlength{\oddsidemargin}{25pt}
\setlength{\textwidth}{420pt}

\makeatletter

% Set the roman font to font for URL (the default font for URL is type writer font).
\renewcommand\UrlFont{\rmfamily}

\newenvironment{Frame}%
{\setlength{\fboxsep}{15pt}
\setlength{\mylength}{\linewidth}%
\addtolength{\mylength}{-2\fboxsep}%
\addtolength{\mylength}{-2\fboxrule}%
\Sbox
\minipage{\mylength}%
\setlength{\abovedisplayskip}{0pt}%
\setlength{\belowdisplayskip}{0pt}%
}%
{\endminipage\endSbox
\[\fbox{\TheSbox}\]}

\newcommand{\bysameline}{\hskip.3em \leavevmode\rule[.5ex]{3em}{.3pt}\hskip0.5em}

%% Define \BibTeX command
\def\BibTeX{{\rm B\kern-.05em{\sc i\kern-.025em b}\kern-.08em
    T\kern-.1667em\lower.7ex\hbox{E}\kern-.125emX}}
\makeatother

%%% title, author, acknowledgement, and date
\title{\texttt{econ.bst}:\\
\BibTeX{} style for economics\\(ver. 3.1)\thanks{\texttt{econ.bst} is available at
\url{https://github.com/ShiroTakeda/econ-bst}}
}

\author{Shiro Takeda\thanks{email: {\ttfamily shiro.takeda@gmail.com}.}}
\date{2021-05-23}

%#####################################################################
%######################### Document Starts ###########################
%#####################################################################
\begin{document}

%%% Title
\maketitle

\newlength{\mylength}
\setlength{\fboxsep}{15pt}
\setlength{\mylength}{\linewidth}
\addtolength{\mylength}{-2\fboxsep}
\addtolength{\mylength}{-2\fboxrule}

%%% Table of contents.
\tableofcontents

%########################## Text Starts ##############################

\setlength{\baselineskip}{16pt}

\section{Main features}

``\texttt{econ.bst}'' is a \BibTeX{} style file. It provids the following
features:
\begin{Frame}
 \begin{itemize}
  \item The author-year type citation (you need ``\texttt{natbib.sty}'' as well).
  \item Reference style used in economics papers.
  \item Highly customizable.  You can easily customize reference style as you
        wish.
  \item You can use ``\textit{certified random order}'' proposed by
        \href{http://dx.doi.org/10.1257/aer.20161492}{\citet[][AER]{10.1257/aer.20161492}}.
        \begin{itemize}
         \item If you want to use certified random order, please see Section
               \ref{sec:cro}.
        \end{itemize}
 \end{itemize}
\end{Frame}

\textbf{The third feature is the key characteristic of ``\texttt{econ.bst}''.}

\section{How to use ``\texttt{econ.bst}''}

\begin{itemize}
 \item ``\texttt{econ.bst}'' requires ``\texttt{natbib.sty}'', which is
       installed in the standard \TeX{} system.
 \item Put ``\texttt{econ.bst}'' file somewhere under the directory (folder)
       ``\texttt{/texmf/bibtex/bst}''. Or you can place ``\texttt{econ.bst}'' at
       the same directory as the \TeX{} file you are editting.
 \item See ``\texttt{econ-example.tex}'' for preparing an tex file.
 \item For \BibTeX{} in general, see, for example,
       \href{https://www.economics.utoronto.ca/osborne/latex/BIBTEX.HTM}{Using
       BibTeX: a short guide} by Martin J. Osborne.
\end{itemize}


\subsection{Preparing a tex file}

In the tex file you are editting, you first need to load \texttt{natbib}
package in the preample (i.e. before \verb|\begin{document}|)
\begin{Frame}
\begin{verbatim}
\usepackage{natbib}
\end{verbatim}
\end{Frame}

In the main text, you can cite bibliographic entries by \verb|\citet|
command.

For example, ``\verb|\citet{Takeda2014a}| is my paper'' generates the output like
\begin{Frame}
\citet{Takeda2014a} is my paper.
\end{Frame}

At the place where you want to add ``the reference part'', you have to set the
bibliography style (\texttt{econ.bst}) and database (\texttt{econ-example.bib}) like
\begin{Frame}
\begin{verbatim}
\bibliographystyle{econ}
\bibliography{econ-example}
\end{verbatim}
\end{Frame}
Note that you should not attach the extension ``bst'' and ``bib'' to the bst and
database file name.

\subsection{Running pdf\TeX{}}

\texttt{econ-example.tex} is created so that it is compiled with pdf\LaTeX
(``\texttt{pdflatex.exe}''). So to compile \texttt{econ-example.tex}, you just
execute the following commnads.
\begin{Frame}
\begin{verbatim}
pdflatex econ-example.tex
bibtex econ-example
pdflatex econ-example.tex
pdflatex econ-example.tex
\end{verbatim}
\end{Frame}

\section{Customization}

``\texttt{econ.bst}'' defines many functions which have names like
``\texttt{bst.xxx.yyy}''.  You can easily customize the reference style by
changing the contents of these functions.

\subsection{Notes on customization}

\begin{itemize}
 \item Customization here is customization of the reference part.  Style
       in the citation part mainly depends on a style file for citation
       (``\texttt{natbib.sty}'').
 \item Except for some cases, customization here cannot change order
       of fields (order of author, year, title etc.)
 \item Functions with ``\texttt{.pre}'' define strings attached to the start
       of the field and functions with ``\texttt{.post}'' define strings
       attached to the end of field.  For exmple,
       ``\texttt{bst.author.pre}'' defines strings attached before author.
 \item You can change order of entries (references).  It will be
       explained in Section \ref{sec:sort_rule}.
 \item In a bst file, integer (e.g. 0, 1, 2) is represented as \verb|#0|, \verb|#1| , \verb|#2|.
\end{itemize}

\subsection{Customized bst files}

Many customized bst files listed in Table \ref{tab:c-files} are already
provided. You can find these files under the ``customization'' folder.  See
\url{https://github.com/ShiroTakeda/econ-bst/tree/master/customization}.

\begin{table}[htbp]
\begin{flushleft}
\caption{List of customized files.}
\label{tab:c-files} 
\begin{tabular}{l|l} \hline \hline
\multicolumn{1}{c}{bst files} & \multicolumn{1}{c}{Explanation} \\\hline
\texttt{econ.bst}              & The default econ.bst file. \\\hline
\texttt{econ-a.bst}            & A simple style. \\\hline
\texttt{econ-b.bst}            & A style with much decoration. \\\hline
\texttt{econ-no-sort.bst}      & This style lists entries in citation order. \\\hline
\texttt{econ-abbr.bst}         & This style uses abbreviated journal name. \\\hline
\texttt{econ-aea.bst}          & The style for AEA journals such as AER, JEL, AEJ and JEP. \\\hline
\texttt{econ-econometrica.bst} & The style for Econometrica. \\\hline
\texttt{econ-jpe.bst}          & The style for JPE (\textit{Journal of Poilitical Economy}). \\\hline
\texttt{econ-jet.bst}          & The style for JET (\textit{Journal of Economic Theoriy}). \\\hline
\texttt{econ-jie.bst}          & The style for JIE (\textit{Journal of International Economics}). \\\hline
\texttt{econ-old.bst}          & The style of the old econ.bst. \\\hline
\end{tabular}
\end{flushleft}
\end{table}

\subsection{Examples of customization}

\subsubsection{Change delimiter used to separate mutiple author names
   from ``and'' to ``\&''.}

For this, change the contents of ``\texttt{bst.and}'' and ``\texttt{bst.ands}''.
\paragraph{Default definition:}
\begin{Frame}
\begin{verbatim}
FUNCTION {bst.and}
{ " and " }
FUNCTION {bst.ands}
{ ", and " }
\end{verbatim}
\end{Frame}

\paragraph{New definition:}
\begin{Frame}
\begin{verbatim}
FUNCTION {bst.and}
{ " \& " }
FUNCTION {bst.ands}
{ " \& " }
\end{verbatim}
\end{Frame}

Then, author names in reference part are displayed as follows:
\begin{center}
Fujita, Masahisa, Paul~R. Krugman, and Anthony~J. Venables \\
 $\downarrow$ \\
Fujita, Masahisa, Paul~R. Krugman \& Anthony~J. Venables 
\end{center}

See ``\texttt{econ-b.bst}'' for an example.

\subsubsection{Make author to small caps style}

For this, change the contents of ``\texttt{bst.author.pre}'' and ``\texttt{bst.author.post}''.
\paragraph{Default definition:}
\begin{Frame}
\begin{verbatim}
FUNCTION {bst.author.pre}
{ "" }
FUNCTION {bst.author.post}
{ "" }
\end{verbatim}
\end{Frame}

\paragraph{New definition:}
\begin{Frame}
\begin{verbatim}
FUNCTION {bst.author.pre}
{ "\textsc{" }
FUNCTION {bst.author.post}
{ "}" }
\end{verbatim}
\end{Frame}

Then, author names in reference part are changed as follows:
\begin{center}
Fujita, Masahisa, Paul~R. Krugman, and Anthony~J. Venables \\
 $\downarrow$ \\
\textsc{Fujita, Masahisa, Paul~R. Krugman, and Anthony~J. Venables}
\end{center}

See ``\texttt{econ-b.bst}'' for an example.

\subsubsection{Change the style of volume and number}

For this, change the contents of ``\texttt{bst.volume.pre}'',
``\texttt{bst.volume.post}'', ``\texttt{bst.number.pre}'' and
``\texttt{bst.number.post}''.  

\paragraph{Default definition:}
\begin{Frame}
\begin{verbatim}
FUNCTION {bst.volume.pre}
{ ", " }
FUNCTION {bst.volume.post}
{ "" }
FUNCTION {bst.number.pre}
{ " (" }
FUNCTION {bst.number.post}
{ ")" }
\end{verbatim}
\end{Frame}

\paragraph{New definition:}
\begin{Frame}
 \begin{verbatim}
FUNCTION {bst.volume.pre}
{ ", Vol. " }
FUNCTION {bst.volume.post}
{ "" }
FUNCTION {bst.number.pre}
{ ", No. " }
FUNCTION {bst.number.post}
{ "" }
\end{verbatim}
\end{Frame}

By this, the style of volume and number change from ``5 (10)'' to ``Vol. 5,
Non. 10''.  See ``\texttt{econ-b.bst}'' for an example.

\subsubsection{Abbreviation of author name}

By default, when there are mutiple documents of the same author, author
name except for the first document is abbreviated by \verb|\bysame|
command (i.e. \bysameline).

If you want to always show author name for all documents, change the
content of ``\texttt{bst.use.bysame}'' as follows:
\begin{Frame}
\begin{verbatim}
FUNCTION {bst.use.bysame}
{ #0 }  
\end{verbatim}
\end{Frame}

In the default setting (`\texttt{bst.use.bysame}' is set to \#1), author names
are abbreviated when they are exactly the same. For example, suppose that there
are the following entries
\begin{itemize}
 \item Mazda, A., Subaru, B., and Honda, C., (2011) "ABC"
 \item Mazda, A., Subaru, B., and Honda, C., (2011) "DEF"
 \item Mazda, A., Subaru, B., and Toyota, D., (2011) "GHI"
\end{itemize}
\vspace*{1em}

In the default setting, these entries are listed like
\begin{Frame}
\begin{itemize}
 \item Mazda, A., Subaru, B., and Honda, C., (2011) "ABC" 
 \item \bysameline, (2011) "DEF"
 \item Mazda, A., Subaru, B., and Toyota, D., (2011) "GHI"
\end{itemize}
\end{Frame}

That is, the abbreviation of authors by `bysame' is only applied to
entries with exactly the same authors.

If you set \#2 to `bst.use.bysame' like
\begin{Frame}
\begin{verbatim}
FUNCTION {bst.use.bysame}
{ #2 }  
\end{verbatim}
\end{Frame}
you can choose alternative abbreviation style like
\begin{Frame}
\begin{itemize}
 \item Mazda, A., Subaru, B., Honda, C., (2011) "ABC"
 \item \bysameline, \bysameline, and \bysameline, (2011) "DEF"
 \item \bysameline, \bysameline, and Toyota, D., (2011) "GHI"
\end{itemize}
\end{Frame}
\vspace*{1em}

See ``\texttt{econ-a.bst}'' and ``\texttt{econ-b.bst}'' for examples.


\subsubsection{Order of first and last name in author field}

``\texttt{bst.author.name}'' defines order of first and last name in author field.

\paragraph{Default definition:}
\begin{Frame}
\begin{verbatim}
FUNCTION {bst.author.name}
{ #0 }
\end{verbatim}
\end{Frame}

If you change \verb|#0| to \verb|#1| or \verb|#2|, you can customize
order of first and family name.  For example, suppose author field is
defined as follows:
\begin{verbatim}
  author =       {Masahisa Fujita and Paul R. Krugman and Anthony J. Venables}
\end{verbatim}

According to the content of ``\texttt{bst.author.name}'', expression of
author changes as follows:
\begin{Frame}
\begin{itemize}
 \item \verb|#0|: First author $\rightarrow$ last-first, other authors
       $\rightarrow$ first-last.\\
       $\rightarrow$ Fujita, Masahisa, Paul~R. Krugman, and
       Anthony~J. Venables
 \item \verb|#1|: All authos $\rightarrow$ last-first \\
       $\rightarrow$ Fujita, Masahisa, Krugman, Paul~R., and Venables, Anthony~J.
 \item \verb|#2|: All authors $\rightarrow$ first-last \\
       $\rightarrow$ Masahisa Fujita, Paul~R. Krugman, and Anthony~J. Venables
\end{itemize}
\end{Frame}

\subsubsection{First name in initial}

By default, first name is displayed in full.  If you change the content
of ``\texttt{bst.first.name.initial}'' to non-zero, first name is displayed
in initial.  For example,
\begin{center}
Fujita, Masahisa, Paul~R. Krugman, and Anthony~J. Venables \\
 $\downarrow$ \\
Fujita, M., P.~R. Krugman, and A.~J. Venables
\end{center}

See ``\texttt{econ-a.bst}'' for an exmpale.

\subsubsection{Decapitalize letters in title field}

Suppose that the title field is defined as follows
\begin{center}
  \verb| title =        {Econometric Policy Evaluation: A Critique}|
\end{center}

Then, title is displayed in reference as follows:
\begin{center}
 Econometric Policy Evaluation: A Critique
\end{center}

If you change the content of ``\texttt{bst.title.lower.case}'' to
non-zero, letters except the first letter are decapitalized.  That is,
you get the following expression in reference:
\begin{center}
 Econometric policy evaluation: A critique
\end{center}

See ``\texttt{econ-a.bst}'' for an exmpale.


\subsubsection{Number index before documents in reference}

You can put the number index to each documents as in
``\texttt{plain.bst}''.  For this, change the content of
``\texttt{bst.use.number.index}'' to non-zero.
\begin{Frame}
\begin{verbatim}
FUNCTION {bst.use.number.index}
 { #1 }
\end{verbatim}
\end{Frame}

If you use fonts other than computer modern fonts, you had better adjust
the contents of functions ``\texttt{bst.number.index.xxx.yyy}''.

See ``\texttt{econ-b.bst}'' for an exmpale.


\subsubsection{List old references first}

By default, references written by the same author are listed in
chronological order (old documents are listed first).  If you change the
contents of ``\texttt{bst.reverse.year}'' to non-zero, the order is
reversed.
\begin{Frame}
\begin{verbatim}
FUNCTION {bst.reverse.year}
{ #1 }
\end{verbatim}
\end{Frame}

\subsubsection{Change the position of year}

By default, year is always displayed right after author name.  You can
change the place of year by setting other values to
``\texttt{bst.year.position}''.
\\

If non-zero set to ``\texttt{bst.year.position}'', year is placed
\begin{itemize}
 \item at the end of line if there is no ``\texttt{note}'' field, 
 \item and before ``\texttt{note}'' field if there is.
\end{itemize}
for non-article type entry.

With respect to aritcle type entry, the following rule is applied:
\begin{Frame}
\begin{enumerate}
 \item \verb|#1|: $\rightarrow$ year is placed at the end (before note field).
 \item \verb|#2|: $\rightarrow$ year is placed after journal name.
 \item \verb|#3|: $\rightarrow$ year is placed after volume.
\end{enumerate}
\end{Frame}

For example, reference style changes as follows:
\begin{description}
 \item[\#0: ] Mazda, A. and B. Subaru, 2007, ``ABC,'' \textit{Journal of
            Automobiles}, 1 (2), pp. 1-10.
 \item[\#1: ] Mazda, A. and B. Subaru, ``ABC,'' \textit{Journal of Automobiles},
            1 (2), pp. 1-10, 2007.
 \item[\#2: ] Mazda, A. and B. Subaru, ``ABC,'' \textit{Journal of Automobiles},
            2007, 1 (2), pp. 1-10.
 \item[\#3: ] Mazda, A. and B. Subaru, ``ABC,'' \textit{Journal of Automobiles},
            1 (2), 2007, pp. 1-10.
\end{description}

See ``\texttt{econ-a.bst}'' and ``\texttt{econ-b.bst}'' for exmpales.


\subsubsection{Order of editors and book title in incollection and inproceedings entry}

By default, editor name comes before book title in incollection and
inproceedings entry. You can reverse this order by
setting non-zero to `bst.editor.btitle.order' like
\begin{Frame}
\begin{verbatim}
FUNCTION {bst.editor.btitle.order}
{ #1 }
\end{verbatim}
\end{Frame}


\begin{Frame}
Krugman, Paul~R. (1991) ``Is Bilateralism Bad?'' in Elhanan Helpman and Assaf
  Razin eds.  {\it International Trade and Trade Policy}, Cambridge, MA: MIT
  Press, pp. 9--23.
\begin{center}
 $\downarrow$
\end{center} 
Krugman, Paul~R. (1991) ``Is Bilateralism Bad?'' in  {\it International Trade
and Trade Policy}  eds. by Elhanan Helpman and Assaf Razin, Cambridge, MA:
MIT Press, pp. 9--23.
\end{Frame}

\subsubsection{Order of address and publisher}

By default, publisher address is placed before publisher name.  You can
reverse this order by setting non-zero to `bst.address.position'.
\begin{Frame}
Krugman, Paul~R. (1991) ``Is Bilateralism Bad?'' in Elhanan Helpman and
  Assaf Razin eds.  {\it International Trade and Trade Policy},
  Cambridge, MA, MIT Press, pp. 9--23.
\begin{center}
 $\downarrow$
\end{center} 
Krugman, Paul~R. (1991) ``Is Bilateralism Bad?'' in Elhanan Helpman and Assaf
  Razin eds.   {\it International Trade and Trade Policy}, MIT Press,
  Cambridge, MA, pp. 9--23.
\end{Frame}

\subsubsection{When the number of authors is very large}

In some research areas, the number of authors of a paper is sometimes very large
(more than 100 ). In such a bibliographic entry, it is not possible to list all
authors names in the reference part.  To handle a bibliographic entry with many
authors, ``\texttt{econ.bst}'' lists only a part of authors when the number of
authors is large. In particular, econ.bst works as follows:
\begin{itemize}
 \item If the number of authors is greater than N1, only the first N2 authors'
       names are displayed in the reference part (and other authors' names are
       omitted by ``et el.'').
 \item N1 is determined by `bst.max.author.num' (the default value = 8).
 \item N2 is determined by `bst.max.author.num.displayed' (the default value =
       3).
\end{itemize}

This function is applied to The following three entries:
\citet{essd-10-405-2018}, \citet{luthi08:_high}, and
\citet{doi:10.1175/2009BAMS2778.1}. In addition, \texttt{econ-many-authors.tex}
in the customization folder includes other examples.


\subsection{Customization of citation part}

We have so far seen the customization of reference part. ``\texttt{econ.bst}''
also can change the styles of citation part although it is highly limited.

\subsubsection{The number of authors for ``\textit{et al.}'' in citation part}

With \texttt{natbib.sty}, you can cite a paper by both full names and
abbreviated names. For example, by using command `\verb|\citet|' and
`\verb|\citet*|', you can generate the following two styles.
\begin{itemize}
 \item Citation by full names: \\
       \verb|\citet*{Takeda2014a}| $\longrightarrow$ \citet*{Takeda2014a}
 \item Citation by abbreviated names: \\
       \verb|\citet{Takeda2014a}| $\longrightarrow$ \citet{Takeda2014a}
\end{itemize}

By the default, author names more than three are abbreviated. But you can change
the number of authors to which abbreviation is applied.  For example, if you set
\verb|#4| to `\texttt{bst.and.others.num}`, abbreviation is applied only to the
entry with more than four authors.
\begin{Frame}
\begin{verbatim}
FUNCTION {bst.and.others.num}
{ #4 }
\end{verbatim} 
\end{Frame}

\subsection{Customization of citation part by \texttt{natbib.sty}}

The style of citation part is mainly controlled by \texttt{natbib.sty}. You can
change the style of citation part by changing commands and options provided by
\texttt{natbib.sty}.
See the manual of \texttt{natbib.sty} for the details. 
\begin{itemize}
 \item The manual of \texttt{natbib.sty}:
       \href{https://mirror.ctan.org/macros/latex/contrib/natbib/natbib.pdf}{natbib.pdf}
 \item The reference sheet of \texttt{natbib.sty}:
       \href{https://mirror.ctan.org/macros/latex/contrib/natbib/natnotes.pdf}{natnotex.pdf}
\end{itemize}

\subsubsection{Citation by number index}

Here I show one customization of \texttt{natbib.sty} for citation style.  With
\texttt{natbib.sty}, we usually use ``author (year)'' citation format. However,
we can use citation by numder index like [1], [2], [3]...

To use this style, you first need to load \texttt{natbib} package with option
``\texttt{numbers}''
\begin{Frame}
\begin{verbatim}
\usepackage[numbers]{natbib}        
\end{verbatim} 
\end{Frame}

In addition, you need to use \texttt{citep} command in the citation part like
\begin{Frame}
\begin{verbatim}
\citep{Takeda2014a}
\end{verbatim} 
\end{Frame}

See ``\texttt{econ-numbers.tex}'' and ``\texttt{econ-numbers.pdf}'' for an example.

\section{Certified random order}
\label{sec:cro}

\subsection{What is ``certified random order''?}

The ordering of author names in joint publications in economics is typically
alphabetical.  However, alphabetical order has its shortcomings; in particular,
it confers greater benefits on the first author.\footnote{See
\citet{10.1257/aer.20161492} for the details.} This can happen for various
reasons, including the use of the ``et al.'' convention, the bunching of
bibliographic references for early-ordered names in reference lists, and the
psychological focus on first authors.

\citet{10.1257/aer.20161492} propose ``certified random order,'' in which names
are separated by a symbol, \textcircled{r}, that certifies or signals that the
author names are in random order. In particular, the symbol permits authors to
``certify'' that a reverse name order is only due to randomization, and avoids
the usual interpretation of name reversal in economics.

``\texttt{econ.bst}'' can handle references to papers in random order. Examples
of such papers in this document are \citet{10.1257/aer.20161492},
\citet{RePEc:hka:wpaper:2018-037}, \citet{vohra18:_maxim_farsig_stabl_set},
\citet{NBERw25205}.


\subsection{How to cite papers in certified random order}

To cite papers in certified random order with ``\texttt{econ.bst}'', you must
flag those entries in the bib file.  To achieve this, ``\texttt{econ.bst}''
contains the field ``\texttt{nameorder}''. If the bibliographical entry uses
random order, add the following information to that entry:
\begin{verbatim}
    nameorder    = {random}
\end{verbatim}

For intance, the entry for \citet{10.1257/aer.20161492} is specified in
\texttt{econ-example.bib} as follows:
\begin{verbatim}
@article{10.1257/aer.20161492,
  Author       = {Ray, Debraj and Robson, Arthur},
  Title        = {Certified Random: A New Order for Coauthorship},
  Journal      = {American Economic Review},
  Volume       = {108},
  Number       = {2},
  Year         = {2018},
  Month        = {February},
  Pages        = {489-520},
  DOI          = {10.1257/aer.20161492},
  URL          = {http://www.aeaweb.org/articles?id=10.1257/aer.20161492},
  nameorder    = {random}
}
\end{verbatim}

\noindent \textbf{[Notes]} \\
1. The field ``nameorder'' is currently unique to \texttt{econ.bst} and it
is not a field used in bibliography databases (at least for now).

\noindent 2. ``\texttt{econ.bst}'' does not itself randomize the order of
names. It simply cites an existing reference in random order, as the authors
intend it to be cited.
\vspace*{1em}


With the ``nameorder'' field in place, there is nothing else you need to change,
and you can cite bibliographic entries as before.  For example, the command
\verb|\citet{10.1257/aer.20161492}| will generate the citation
\citet{10.1257/aer.20161492} in the text, and the following entry in the
references:

\begin{quote}
Ray, Debraj \textcircled{r} Arthur Robson (2018) ``Certified Random: A New
Order for Coauthorship,'' \textit{American Economic Review}, 108 (2),
489--520, \href{http://dx.doi.org/10.1257/aer.20161492}{\urlstyle{rm}
\nolinkurl{10.1257/aer.20161492}}.
\end{quote}

Under the random order style, an entry with two authors is cited as already
described: \citet{10.1257/aer.20161492}. On the other hand, an entry with more
than two authors has two citation styles, depending on whether all names are
cited in full, or the citation is abbreviated:
\begin{itemize}
 \item All-names citation $\longrightarrow$ \citet*{NBERw25205}
 \item Abbreviated citation $\longrightarrow$ \citet{NBERw25205}
\end{itemize}

Note that in the abbreviated names style, \textcircled{r} is used instead of
``et''. Think of \textcircled{r} as a conjunction just as ``and'' or ``but''. In
verbal citations, it would be pronounced just as the letter ``r'' is pronounced.

\subsection{Customization for certified random order}

The behavior of random order is controlled by the following functions:
`\texttt{bst.use.nameorder}', `\texttt{bst.and.nameorder}',
`\texttt{bst.cite.and.nameorder}' and `\texttt{bst.and.others.nameorder}'.
\vspace*{1em}

Under the default setting of ``\texttt{econ.bst}'', random order is
automatically used for bibliographic entries with `\texttt{nameorder}' field.
However, if you set \verb|#0| to `\texttt{bst.use.nameorder}' as follows, the
function of random order is not used.
\begin{Frame}
\begin{verbatim}
FUNCTION {bst.use.nameorder}
{ #0 }  % Not use random order
\end{verbatim}
\end{Frame}

Under the default setting, we use ``\textcircled{r}'' (\verb|\textcircled{r}|)
for the symbol for separating author names.
They are defined with the following code in \texttt{econ.bst}.
\begin{Frame}
\begin{verbatim}
FUNCTION {bst.and.nameorder}
{ " \textcircled{r} " }

FUNCTION {bst.cite.and.nameorder}
{ " \textcircled{r} " }

FUNCTION {bst.and.others.nameorder}
{ " \textcircled{r}~al." }
\end{verbatim}
\end{Frame}

You can use an alternative symbol by redefining these functions:
`\texttt{bst.and.nameorder}', `\texttt{bst.cite.and.nameorder}' and
`\texttt{bst.and.others.nameorder}'.


\section{Sorting rule}
\label{sec:sort_rule}

\noindent \textbf{[Note]} If you want to create an ordinary list of
references, you need not  read this part.  The explanation below is
for sorting references in a special way.

\subsection{Basic sorting rule}

The sorting of references is done according to values of fields defined
in bib files.  Basically, sorting is done according to the following
order of priority:
\begin{enumerate}
 \item Type of entry (if ``\texttt{bst.sort.entry.type}'' has non-zero value)
 \item Value of ``\texttt{year}'' field (if ``\texttt{bst.sort.year}'' has non-zero value)
 \item Value of ``\texttt{absorder}'' field (if ``\texttt{absorder}'' field has
       a value)
 \item Value of ``\texttt{author}'' (or ``\texttt{editor}'') field.
       \begin{itemize}
        \item If ``\texttt{sortname}'' field is assigned a value, it is used
              instead of ``\texttt{author}'' and ``\texttt{editor}''.
       \end{itemize}
 \item Value of ``\texttt{year}'' field.
 \item Value of ``\texttt{order}'' field.
 \item Value of ``\texttt{month}'' field.
 \item Value of ``\texttt{title}'' field.
\end{enumerate}

By default, 
\begin{itemize}
 \item ``\texttt{bst.sort.entry.type}'' and ``\texttt{bst.sort.year}'' have
       zero,
 \item \textbf{``\texttt{absorder}''} and \textbf{``\texttt{order}''}
       fields are not assigned values because they are fields specific
       to ``\texttt{econ.bst}''.
\end{itemize}
Thus, references are sorted according to
\begin{Frame}
 \begin{center}
 ``\texttt{author}'' $\rightarrow$ ``\texttt{year}'' $\rightarrow$
 ``\texttt{month}'' $\rightarrow$  ``\texttt{title}'' 
\end{center}
\end{Frame}
That is, ``\texttt{author}'' is used as the primary key, ``\texttt{year}'' as
the secondary key, ``\texttt{month}'' as the third key and ``\texttt{title}'' as
the fourth key.

\subsection{No sorting}

If you want to list references in citation order, set non-zero value to
``\texttt{bst.no.sort}''.
\begin{Frame}
\begin{verbatim}
FUNCTION {bst.no.sort}
{ #1 }
\end{verbatim}
\end{Frame}

Note that when you set non-zero value to ``\texttt{bst.no.sort}'', you had better not
use \verb|\bysame|.

\subsection{Sort references by type}

If you want to gather references according to their types (article,
book, incollection, unpublished etc.), set non-zero value to
``\texttt{bst.sort.entry.type}''.
\begin{Frame}
\begin{verbatim}
FUNCTION {bst.sort.entry.type}
{ #1 }
\end{verbatim}
\end{Frame}

Order of listing by entry type is determined by function
``\texttt{bst.sort.entry.type.order}'' (by default, listed in alphabetical
order, that is, article $\rightarrow$ book $\rightarrow$ booklet
$\rightarrow$ comment $\rightarrow$ inbook $\rightarrow$ incollection
$\rightarrow$ $\cdots$ $\rightarrow$ unpublished).  See
``\texttt{bst.sort.entry.type.order}'' in ``\texttt{econ.bst}''.

\subsection{Use ``\texttt{year}''  as the primary sorting key}

When you create CV or a list of your papers, you may want to sort
references in chronological order.  If all papers are written soley by
yourself, references are sorted in chronological order by default.
However, there are co-writers and if you are not the first author,
references are not sorted in chronological order because the author name
is used as the primary sorting key by default.
If you want to sort references in chronological order even when
there are co-writers, set non-zero to ``\texttt{bst.sort.year}''.
\begin{Frame}
\begin{verbatim}
FUNCTION {bst.sort.year}
{ #1 }
\end{verbatim}
\end{Frame}

By default, old references are listed first.  But if you set non-zero to
``\texttt{bst.reverse.year}'', new references are listed first.

\subsection{Sorting by ``\texttt{absorder}'' field}

If ``\texttt{absorder}'' is defined in bib file,
``\texttt{econ.bst}'' uses its content as the primary sorting key.
You can set number 0--999 to ``\texttt{absorder}'' field.

\begin{Frame}
\begin{center}
 no absorder or absorder = 0  $\rightarrow$ absorder = 1 $\rightarrow$ absorder = 2
 $\rightarrow$ $\cdots$ $\rightarrow$ absorder = 999
\end{center}
\end{Frame}
That is, reference with a small value of ``\texttt{absorder}'' is listed
first.  In this document (``\texttt{econ-example.bib}''), the reference
with the key \citet{takeda10:_cge_analy_welfar_effec_trade} has 999 for
``\texttt{absorder}'' field and thus listed in the last.


\subsubsection{Ignore ``\texttt{absorder}'' field}

If you set some values for ``\texttt{absorder}'' fields in bib file,
but if you want to ignore them, set non-zero to
``\texttt{bst.notuse.absorder.field}''.
\begin{Frame}
\begin{verbatim}
FUNCTION {bst.notuse.absorder.field}
{ #1 }
\end{verbatim}
\end{Frame}

\subsection{Sorting by ``\texttt{sortname}'' field}

If ``\texttt{sortname}'' field is assigned a value, it is used for sorting
instead of ``\texttt{author}'' or ``\texttt{editor}''.

For example, the first author name of the following bibliography entry starts
with the letter ``d'' (d'Aspremont), so it is listed after the documents with
author names that start with the letter ``c''.
\begin{Frame}
\begin{verbatim}
@article{d'Aspremont-2004-BalancedBayesianMechanisms,
  title        = {Balanced Bayesian Mechanisms},
  author       = {{d'Aspremont}, Claude and Cr{\'e}mer, Jacques and
                  {G{\'e}rard-Varet}, Louis-Andr{\'e}},
  year         = 2004,
  volume       = 115,
  pages        = {385--396},
  issn         = {0022-0531},
  doi          = {10.1016/j.jet.2003.07.001},
  journal      = {Journal of Economic Theory},
  number       = 2,
}
\end{verbatim}
\end{Frame}

However, for the name ``d'Aspremont'', we should rather think that it starts
with the letter ``A''. In this case, you had better to add the following
``\texttt{sortname}'' field
\begin{Frame}
\begin{verbatim}
  sortname     = {Aspremont, Claude and Cr{\'e}mer, Jacques and
                  {G{\'e}rard-Varet}, Louis-Andr{\'e}},
\end{verbatim}
\end{Frame}

Then, this entry is listed as author whose name begin with letter ``A''. The
following three entries have sortname fields:
\citet{d'Aspremont-2004-BalancedBayesianMechanisms},
\citet{d'Aspremont-2003-CorrelationIndependenceAndBayesian}, and
\citet{d'Aspremont-1998-LinearInequalityMethodsTo}.

\section{Misc.}

\begin{itemize}
 \item Email: \verb|<shiro.takeda@gmail.com>|.
 \item ``\texttt{econ.bst}'' is available at \url{https://github.com/ShiroTakeda/econ-bst}.
\end{itemize}
\vspace*{1em}

\citet{borgers95:_note_implem_stron_domin},
\citet{bergemann11:_ration},
\citet{takeda2015a},
\citet{Takeda2014a},
\citet{Biker-2007-unemployment},
\citet{Babiker-1999-JapaneseNuclearPower},
\citet{Babiker-1999-KyotoProtocoland},
\citet{Babiker2000525},
\citet{BabikerRutherford-2005-EconomicEffectsof},
\citet{goldin:katz:2011}, \citet{goldin:katz:2008}, \citet{goldin:katz:2000}.
\citet{stakeda2019web},
\citet{Rivers-2005-CombiningTop-Downand},
\citet{WilsonMannOtsuki-2005-AssessingBenefitsof},
\citet{zhang2016Deep},
\citet{imbens2019Optimized},
\citet{essd-10-405-2018},
\citet{luthi08:_high},
\citet{doi:10.1175/2009BAMS2778.1},
\citet{d'Aspremont-2004-BalancedBayesianMechanisms},
\citet{d'Aspremont-2003-CorrelationIndependenceAndBayesian},
\citet{Chung-1999-ANoteOnMatsushima's},
\citet{d'Aspremont-1998-LinearInequalityMethodsTo},

\vspace*{1em}

\nocite{*}

%%% BibTeX style.
\bibliographystyle{econ}

%% BibTeX database file.
\addcontentsline{toc}{section}{References}
\bibliography{econ-example}

\end{document}

%#####################################################################
%######################### Document Ends #############################
%#####################################################################
% --------------------
% Local Variables:
% fill-column: 80
% coding: utf-8-unix
% End:

